\chapter{Nozioni Fondamentali}

L'obiettivo di questo capitolo è quello di presentare i concetti di base che vengono utilizzati nel seguito di questo
lavoro e, più in generale, nell'ambito dell'ottimizzazione matematica.

\section{Introduzione}
Nel corso della sua esistenza, l’uomo ha sempre dovuto affrontare e risolvere una grande varietà di problemi. Con il
passare del tempo, le nostre capacità si sono evolute e gli strumenti a nostra disposizione sono migliorati,
permettendoci di gestire problemi di complessità sempre maggiore. Di conseguenza, oggi non ci accontentiamo più di
risolvere un problema trovando una soluzione arbitraria, ma aspiriamo ad ottimizzare, cioè a identificare la soluzione
migliore possibile, sulla base di criteri specifici.

L'ottimizzazione è ormai parte integrante di tutti gli ambiti che richiedano di risolvere problemi decisionali. Nella
pratica, risolvere un problema decisionale significa assegnare valori alle variabili che lo caratterizzano. In molti
casi, la scelta di questi valori è limitata da un insieme di vincoli che devono essere soddisfatti. L'obiettivo è quello
di ottimizzare una grandezza specifica. Questa grandezza e i vincoli da soddisfare possono spesso essere rappresentati
come funzioni delle variabili coinvolte, permettendoci di definire il problema utilizzando un modello matematico. La
formulazione di un modello dovrebbe essere sufficientemente complessa da rappresentare accuratamente il problema cui si
riferisce e, allo stesso tempo, abbastanza semplice da renderlo trattabile con gli strumenti risolutivi disponibili.

L'ottimizzazione è spesso un processo iterativo, in cui il modello matematico di riferimento viene continuamente
raffinato, con l'obiettivo di ottenere soluzioni sempre più accurate, in relazione al problema del mondo reale a cui è
associato.

Oggi siamo in grado di ottimizzare problemi che coinvolgono un numero di variabili e di vincoli molto elevato. Tuttavia,
esistono classi di problemi che non posso essere risolti all'ottimo in un tempo ragionevole e che hanno determinato lo
sviluppo di algoritmi euristici, con la finalità di ottenere soluzioni accettabili in tempi contenuti.


\section{Problemi di Ottimizzazione}

un problema di ottimizzazione \( \mathcal{P} \) può essere definito con la formulazione generale
\begin{equation}\label{eq:opt_prob}
    \mathcal{P}\colon
    \begin{cases}
        \text{$\min$ (or $\max$)} & f(\vec{x}) \\
                                  & \mathcal{S} \\
                                  & \vec{x} \in \mathcal{D}
    \end{cases}
\end{equation}
dove
\(
    f(\vec{x})
\)
è una funzione a valori reali nelle variabili
\(
    \vec{x} = [x_1 \, \ldots \, x_n] \in \mathcal{D} = D_1 \times \dots \times D_n,
\)
con
\(
    x_j \in D_j \;\; \forall j \colon 1 \leq j \leq n
\)
e
\(
    \mathcal{S}
\)
è un insieme finito di vincoli. Formalmente, un vincolo
\(
    c \in \mathcal{S}
\)
è una funzione che coinvolge un sottoinsieme delle variabili del problema e che può assumere i valori vero o falso,
corrispondenti alle condizioni di vincolo soddisfatto o violato, rispettivamente.

Calcolare il massimo di una funzione
\(
    f(\vec{x})
\)
è equivalente a calcolare il minimo della funzione
\(
    -f(\vec{x})
\). Infatti, i due valori coincidono, a meno del segno, e si ottengono nello stesso punto
\(
    \bar{\vec{x}}
\).
Di conseguenza, nel seguito possiamo limitarci a studiare i problemi di minimo senza perdere di generalità. Tutte le
considerazioni saranno valide, eventualmente con opportune modifiche, anche per i problemi di massimo.

\subsection{Restrizioni e Rilassamenti}

\subsection{Ricerca}

\section{Programmazione Lineare}

\subsection{Interpretazione Geometrica}

\subsection{Formulazioni Equivalenti}

\subsection{Algoritmo del Simplesso}

\subsection{Dualità}

\subsection{Rilassamento Lagrangiano}

\section{Programmazione Lineare Intera}
