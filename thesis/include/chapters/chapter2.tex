\chapter{Set-Covering}

In questo capitolo viene presentato il problema del Set-Covering, che è il problema di riferimento per l'algoritmo
sviluppato nel prossimo capitolo.

\section{Formulazione del Problema}

Consideriamo un insieme \( \mathcal{I} \) di \( m \) elementi e una famiglia \( \mathcal{F} = \{F_1, \ldots, F_n\} \) di
sottoinsiemi di \( \mathcal{I} \). Il problema del Set-Covering richiede di trovare il più piccolo insieme
\(
    \mathcal{F}^{\,\text{\raisebox{2pt}{$\star$}}} \!\subseteq \mathcal{F}
\)
in grado di coprire tutti gli elementi in \( \mathcal{I} \). Formalmente, si richiede che
\begin{equation}
    \bigcup_{F_j \, \in \, \mathcal{F^{\,\text{\raisebox{2pt}{$\star$}}}}} \!\!F_j = \mathcal{I}
\end{equation}
Questo problema può essere modellato utilizzando la programmazione lineare intera: introduciamo una variabile binaria
per ciascun elemento \( F_j \in \mathcal{F} \), che rappresenta l'appartenenza di \( F_j \) a
\(
    \mathcal{F}^{\,\text{\raisebox{2pt}{$\star$}}}.
\)
Nello specifico, definiamo
\begin{equation}
    x_j =
    \begin{cases}
        1, & \text{se } F_j \in \mathcal{F}^{\,\text{\raisebox{2pt}{$\star$}}} \\
        0, & \text{altrimenti}
    \end{cases} \qquad \forall j\colon 1 \leq j \leq n.
\end{equation}
A questo punto, la formulazione del problema segue immediatamente:
\begin{equation}\label{eq:scp1}
    \begin{array}{ll}
        \hspace*{0.3cm}\min  \displaystyle\sum_{j=1}^{n} x_j \\[20pt]
        \hspace*{0.3cm}\displaystyle\sum_{j\colon i \,\in\, F_j} \!\! x_j \geq 1 & \forall i \in \mathcal{I} \\[20pt]
        \hspace*{0.3cm}x_j \in \{0, 1\} & \forall j\colon 1\leq j\leq n
    \end{array}
\end{equation}
Si può facilmente osservare che i limiti superiori per le variabili \( x_j \) sono ridondanti, poiché non c'è mai
convenienza nel selezionare la stessa famiglia \( F_j \) più di una volta. Di conseguenza, possiamo rilassare il vincolo
sulle variabili, e assumere semplicemente che siano intere non negative.

Una formulazione equivalente a quella presentata in \ref{eq:scp1}, si ottiene introducendo i
parametri \( a_{ij} \) in modo tale che
\begin{equation}
    a_{ij} =
    \begin{cases}
        1, & \text{se } i \in F_j \\
        0, & \text{altrimenti}
    \end{cases} \qquad \forall i \in \mathcal{I} ,\;\,\forall j\colon 1 \leq j \leq n
\end{equation}
A questo punto, possiamo definire la formulazione alternativa
\begin{equation}\label{eq:scp2}
    \text{SCP}\colon
    \begin{cases}
        \hspace*{0.3cm}\min  \displaystyle\sum_{j=1}^{n} x_j \\[20pt]
        \hspace*{0.3cm}\displaystyle\sum_{j=1}^n a_{ij}\,x_j \geq 1 & \forall i \in \mathcal{I} \\[20pt]
        \hspace*{0.3cm}x_j \geq 0,\; x_j \in \mathbb{Z} & \forall j\colon 1 \leq j \leq n
    \end{cases}
\end{equation}
Questa formulazione è particolarmente utile nel seguito di questo lavoro e può essere scritta in maniera compatta
introducendo una matrice \( \vec{A} = [a_{ij}] \) e i vettori
\[
    \vec{x} = [x_1, \ldots, x_n]^{\tr} \in \mathbb{R}^n, \quad \vec{c} = [1, \ldots, 1]^{\tr} \in \mathbb{R}^n
    ,\quad b = [1, \ldots, 1]^{\tr} \in \mathbb{R}^m.
\]
Si ottiene quindi la formulazione compatta
\begin{equation}\label{eq:scp3}
    \text{SCP}\colon
    \begin{cases}
        \hspace*{0.3cm}\min & \vec{c}^{\tr}\vec{x}\\
        \hspace*{0.3cm}&\vec{A}\vec{x} \geq \vec{b} \\
        \hspace*{0.3cm}&\vec{x} \geq 0,\; x_j \in \mathbb{Z} \quad \forall j\colon 1 \leq j \leq n
    \end{cases}
\end{equation}

\subsection{Rilassamento Lineare}

Il rilassamento lineare del SCP rimuove il vincolo di interezza, e consente alle variabili di assumere valori reali non
negativi. In questo modo, la formulazione \ref{eq:scp3} diventa quella di un problema di programmazione lineare in forma
canonica:
\begin{equation}\label{eq:scp4}
    \begin{array}{ll}
        \min & \vec{c}^{\tr}\vec{x}\\
        &\vec{A}\vec{x} \geq \vec{b} \\
        &\vec{x} \geq 0
    \end{array}
\end{equation}
Naturalmente, una soluzione del rilassamento lineare non è necessariamente ammissibile per il problema di partenza.
Inoltre, per come è definito, il rilassamento lineare non ha nessun significato in relazione al problema del
Set-Covering. Una soluzione in cui una variabile compare con un valore non intero non ha senso, poichè le variabili del
problema iniziale sono state introdotte con un signficato che dipende fortemente dalla presenza del vincolo di
interezza. Infine, anche per il rilassamento lineare rimane valida la considerazione fatta in precedenza, relativamente
al limite superiore per le variabili, che le vincola implicitamente ad assumere valori nell'intervallo \( [0, 1] \).

\subsection{Problema Duale}

\subsection{Rilassamento Lagrangiano}

\section{Esempi e Applicazioni}
